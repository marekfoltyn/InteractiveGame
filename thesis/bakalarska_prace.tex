% options:
% thesis=B bachelor's thesis
% thesis=M master's thesis
% czech thesis in Czech language
% slovak thesis in Slovak language
% english thesis in English language
% hidelinks remove colour boxes around hyperlinks

\documentclass[thesis=B,czech]{FITthesis}[2012/06/26] %documentclass ... typ dokumentu (definovany v souboru .cls)

\usepackage[utf8]{inputenc} % LaTeX source encoded as UTF-8

\usepackage{graphicx} %graphics files inclusion
% \usepackage{amsmath} %advanced maths
% \usepackage{amssymb} %additional math symbols

\usepackage{dirtree} %directory tree visualisation

% % list of acronyms
% \usepackage[acronym,nonumberlist,toc,numberedsection=autolabel]{glossaries}
% \iflanguage{czech}{\renewcommand*{\acronymname}{Seznam pou{\v z}it{\' y}ch zkratek}}{}
% \makeglossaries

\newcommand{\tg}{\mathop{\mathrm{tg}}} %cesky tangens
\newcommand{\cotg}{\mathop{\mathrm{cotg}}} %cesky cotangens

% % % % % % % % % % % % % % % % % % % % % % % % % % % % % % 
% ODTUD DAL VSE ZMENTE
% % % % % % % % % % % % % % % % % % % % % % % % % % % % % % 

\department{Katedra softwarového inženýrství}
\title{Interaktivn{\' i} ovl{\' a}d{\' a}n{\' i} PC hry pomoc{\' i} chytr{\' e}ho telefonu}
\authorGN{Marek} %(křestní) jméno (jména) autora
\authorFN{Foltýn} %příjmení autora
\authorWithDegrees{Marek Foltýn} %jméno autora včetně současných akademických titulů
\supervisor{Ing. Filip K{\v r}ikava, Ph.D.}
\acknowledgements{Chci velmi poděkovat vedoucímu práce Ing. Filipu K{\v r}ikavovi, Ph.D. za p{\v r}íkladné vedení. Dále pak své manželce Veronice Foltýnové za trpělivost a ochotu vytvá{\v r}et prost{\v r}edí vhodné ke tvorbě bakalářské práce, 
rodičům a celé mé rodině za velkou podporu v mnoha směrech. Děkuji také všem, kteří se podíleli na testování hratelnosti.}
\abstractCS{V~několika větách shrňte obsah a přínos této práce v~češtině. Po přečtení abstraktu by se čtenář měl mít čtenář dost informací pro rozhodnutí, zda chce Vaši práci číst.}
\abstractEN{Sem doplňte ekvivalent abstraktu Vaší práce v~angličtině.}
\placeForDeclarationOfAuthenticity{V~Praze}
\declarationOfAuthenticityOption{4} %volba Prohlášení (číslo 1-6)
\keywordsCS{Nahraďte seznamem klíčových slov v češtině oddělených čárkou.}
\keywordsEN{Nahraďte seznamem klíčových slov v angličtině oddělených čárkou.}

\begin{document} % zacatek celeho dokumentu

% \newacronym{CVUT}{{\v C}VUT}{{\v C}esk{\' e} vysok{\' e} u{\v c}en{\' i} technick{\' e} v Praze}
% \newacronym{FIT}{FIT}{Fakulta informa{\v c}n{\' i}ch technologi{\' i}}

\begin{introduction}

%TODO: NAPSAT UVOD % 

\end{introduction}

\chapter{Ovládání PC hry}

% obecné povídání o způsobech ovládání her %

% vývoj mobilních telefonů a rozšiřování jejich schopností %

Mobilní telefony se v dnešní době rozvíjí velmi rychlým tempem. Obsahují mnoho senzorů okolního prostředí: téměř každý nový smarthphone obsahuje dotykový displej, akcelerometr společně s gyroskopem, proximity senzor, vibrační motorek. Dále pak jsou schopny bezdrátově komunikovat pomocí WiFi, bluetooth a dalších technologií.

Stejně jako u rozvoje stolních počítačů se i u mobilních telefonů nabízí otázka, jak všechny tyto nové technologie využít pro větší zážitek z hraní. Stejně jako u 

Velké využití nabízí dotyková obrazovka. Sjednocuje se zde vizuální část hry s ovládáním. Pokud chce například hráč přesunout objekt, jednoduše se jej dotkne prstem a přetáhne. Ve hrách 

% propojení mobilů a PC her %

% výhody oproti tradičním způsobům ovládání %


\chapter{Požadavky}

\chapter{Návrh}

\chapter{Implementace}

\chapter{Testování}

\chapter{Dokumentace}

\begin{conclusion}
	%sem napište závěr Vaší práce
	
	%TODO: možnosti rozšíření %
\end{conclusion}

\bibliographystyle{csn690}
\bibliography{mybibliographyfile}

\appendix

\chapter{Seznam použitých zkratek}
% \printglossaries
\begin{description}
	\item[GUI] Graphical user interface
	\item[XML] Extensible markup language
\end{description}

\chapter{Obsah přiloženého CD}

%upravte podle skutecnosti

\begin{figure}
	\dirtree{%
		.1 readme.txt\DTcomment{stručný popis obsahu CD}.
		.1 exe\DTcomment{adresář se spustitelnou formou implementace}.
		.1 src.
		.2 impl\DTcomment{zdrojové kódy implementace}.
		.2 thesis\DTcomment{zdrojová forma práce ve formátu \LaTeX{}}.
		.1 text\DTcomment{text práce}.
		.2 thesis.pdf\DTcomment{text práce ve formátu PDF}.
		.2 thesis.ps\DTcomment{text práce ve formátu PS}.
	}
\end{figure}

\end{document}
