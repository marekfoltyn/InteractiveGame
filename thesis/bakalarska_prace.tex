% options:
% thesis=B bachelor's thesis
% thesis=M master's thesis
% czech thesis in Czech language
% slovak thesis in Slovak language
% english thesis in English language
% hidelinks remove colour boxes around hyperlinks

\documentclass[thesis=B,czech,hidelinks]{FITthesis}[2012/06/26] %documentclass ... typ dokumentu (definovany v souboru .cls)

\usepackage[utf8]{inputenc} % LaTeX source encoded as UTF-8

\usepackage{graphicx} %graphics files inclusion
\usepackage{adjustbox}
\usepackage{tabularx}
% \usepackage{amsmath} %advanced maths
% \usepackage{amssymb} %additional math symbols

\usepackage{dirtree} %directory tree visualisation

% % list of acronyms
% \usepackage[acronym,nonumberlist,toc,numberedsection=autolabel]{glossaries}
% \iflanguage{czech}{\renewcommand*{\acronymname}{Seznam pou{\v z}it{\' y}ch zkratek}}{}
% \makeglossaries

\newcommand{\tg}{\mathop{\mathrm{tg}}} %cesky tangens
\newcommand{\cotg}{\mathop{\mathrm{cotg}}} %cesky cotangens

% % % % % % % % % % % % % % % % % % % % % % % % % % % % % % 
% ODTUD DAL VSE ZMENTE
% % % % % % % % % % % % % % % % % % % % % % % % % % % % % % 

\department{Katedra softwarového inženýrství}
\title{Interaktivn{\' i} ovl{\' a}d{\' a}n{\' i} PC hry pomoc{\' i} chytr{\' e}ho telefonu}
\authorGN{Marek} %(křestní) jméno (jména) autora
\authorFN{Foltýn} %příjmení autora
\authorWithDegrees{Marek Foltýn} %jméno autora včetně současných akademických titulů
\supervisor{Ing. Filip K{\v r}ikava, Ph.D.}
\acknowledgements{Chtěl bych poděkovat vedoucímu práce Ing. Filipu K{\v r}ikavovi, Ph.D. za~pomoc a~p{\v r}íkladné vedení práce. Dále pak své manželce Veronice Foltýnové za~trpělivost a~ochotu vytvá{\v r}et prost{\v r}edí vhodné ke~tvorbě bakalářské práce, 
rodičům a~celé mé rodině za~velkou podporu ve~všech směrech. Děkuji také všem, kteří se podíleli na~testování hratelnosti hry.}
\abstractCS{Tato bakalářská práce se zabývá tvorbou systému pro~ovládání PC hry pomocí mobilního telefonu. Hlavním cílem je obohacení herního zážitku pomocí interaktivních prvků, které jsou na~mobilních telefonech k~dispozici. Součástí práce je analýza způsobů ovládání her, přehled interaktivních technologií v mobilních telefonech a~samotná tvorba komunikačního systému, který je demonstrován na~jednoduché hře. }
\abstractEN{The main purpose of this thesis is to create an interactive PC game control system using smartphones in order to enhance the game experience. The thesis contains the analysis of different ways how a PC game can be controlled, the overwiev of interactive mobile technologies and also the communication system implementation, which is demonstrated in a simple game. }
\placeForDeclarationOfAuthenticity{V~Praze}
\declarationOfAuthenticityOption{1} %volba Prohlášení (číslo 1-6)
\keywordsCS{interaktivní ovládání PC hry, smartphone, komunikační systém, Cocos2d-x, RakNet}
\keywordsEN{interactive PC game controller, smartphone, communication system, Cocos2d-x, RakNet}

\begin{document} % zacatek celeho dokumentu

% \newacronym{CVUT}{{\v C}VUT}{{\v C}esk{\' e} vysok{\' e} u{\v c}en{\' i} technick{\' e} v Praze}
% \newacronym{FIT}{FIT}{Fakulta informa{\v c}n{\' i}ch technologi{\' i}}

\begin{introduction}

Počítačové hry existují od~počátku prvních počítačů \cite{rylich}. Jejich možnosti se vyvíjí podobně, jako se vyvíjí výpočetní a~grafický výkon, hardware a~další technologie. Neustálé zmenšování součástek v~současné době nabízí vysoký výkon ve~velmi malých strojích: notebooky, chytré telefony a~dokonce i~hodinky s~vícejádrovými procesory. \cite{kupi}

Na~zmenšování hardwaru se adaptovaly také hry, které se v~hojné míře objevují i~na~přenosných zařízeních, jako jsou mobilní telefony, tablety a~další. Lidé tak mohou kromě počítačového stolu hrát doslova kdekoli.

Spolu s~vývojem počítačů se mění i~způsoby, jak lze počítačové hry ovládat. Kromě tradiční myši a~klávesnice lze v dnešní době využít joystick, volant, gamepad a~jiná podobná zařízení. Všechny tyto technologie se v herním průmyslu snaží obohatit hráčův zážitek intuitivním ovládáním. Existují však elektronická zařízení s~velkým množstvím senzorů, u~kterých se přímo nabízí otázka, jak tyto senzory využít pro~ovládání hry, jsou to mobilní telefony.

Mobilní telefony se v~dnešní době rozvíjí velmi rychlým tempem. Téměř každý nový smarthphone je vybaven dotykovým displejem, akcelerometrem společně s~gyroskopem, proximity senzorem, vibračním motorkem a~dalšímy senzory okolního prostředí. Dále pak jsou schopny bezdrátově komunikovat pomocí WiFi, bluetooth a~při tomto množství interaktivních prvků v~jediném zařízení se přirozeně nabízí otázka, jak všechny tyto nové technologie využít pro~větší zážitek z~hraní. Velké využití nabízí například dotyková obrazovka. Sjednocuje se zde vizuální část hry s~ovládáním. Pokud chce například hráč přesunout objekt, jednoduše se jej dotkne prstem a přetáhne. Všech těchto výhod široce využívají mobilní hry.

V~této práci se budu zabývat hledáním způsobu, jak využít interaktivní prvky mobilních telefonů pro~ovládání počítačové hry. Smartphone tedy nebude sloužit jako samostatná herní konzole, ani jako simulace periferie typu myš nebo gamepad, ale bude tvořit jednotný celek společně se~samostatným počítačem. Tuto myšlenku se budu snažit demonstrovat vytvořením systému komunikace mezi telefony a~počítačem a~jeho využitím v~jednoduché hře. 


\end{introduction}

\chapter{Ovládání her}

Ovládání počítačových her úzce souvisí se~samotným vývojem hardwaru a~především počítačových periferií. V~následující kapitole se budu zabývat uvedením do~problematiky ovládání her v~současné době a~to jak na~počítačích, tak i~na~dalších zařízeních. Nebudu se zde zabývat softwarovým návrhem uživatelského ovládání, ale spíše využitím hardwaru pro herní účely. Hlavní náplní bude srovnání několika rozdílných způsobů ovládání, jejich přínosů a~nevýhod. 

%https://books.google.cz/books?id=hWSUAgAAQBAJ&pg=PT111&dq=game+controller+typology&hl=cs&sa=X&ved=0ahUKEwjumM6j65jMAhXFNpoKHcHEAeQQ6AEIJzAB#v=onepage&q=game%20controller%20typology&f=false

\section{Historie ovladačů}

Nejprve se stručně zaměřím na~historii vývoje hardwaru pro~ovládání her. Počítačové hry jako takové se začaly objevovat v~50. letech 20. století. Hardware dostupný v~této době nebyl určený pro~herní zaměření, např. známá hra \textit{Tennis for Two} vytvořená Williamem Higinbothamem v~roce 1958 v~národní laboratoři v~Brookhavenu využívala osciloskop jako grafický displej pro~zobrazení primitivní dvourozměrné hry a~jako ovládání sloužilo jedno tlačíko a~otočný regulátor.\cite{gamevshardware}

Větší rozvoj herního hardwaru začal v~roce 1971. Byl vytvořen první herní automat na~mince s~hrou \textit{Galaxy Game}. Jednalo se o~hru pro~dva hráče ovládánou jednoduchým joystickem. Později se začaly objevovat další automaty, využívaly jak tlačítka a~joystick, tak i~volant.

Hry se tedy objevovaly především na~herních konzolích. Vzhledem k~narustajícím prodejům osobních PC se hry dostávaly i~do~této oblasti, kde byla nejrozsířenejší periferií klávesnice. Vývoj herních konzolí se však nezastavil.

Převrat v~ovládání PC, a~to i~v~herním průmyslu, způsobil příchod počítačové myši, jak ji známe dnes. Mnoho herních konceptů bylo vylepšeno a~pro~hráče to představovalo větší pohodlí při~hraní. \cite{gamevshardware} Kromě toho se taky vyvíjely alternativní typy ovládání, některé úzce spojeny s herním žánrem (joystick, volant), nebo naopak vhodné pro velké množství her (gamepad).

Kromě klasických ovladačů se začaly objevovat také snahy o~přirozenější ovládání pomocí pohybu, tzv. \textit{motion sensing}. Vzniklo proto několik ovladačů a~herních konzolí, z nichž nejvýznamější je Microsoft Kinect vzniklý v~roce 2010.\cite{wikicontrollers} Více se problematice ovládání pohybem věnuji v sekci \ref{section:motion_capture}.

V~současné době se využívá široká škála způsobu a technologií ovládání. Následující kapitoly budou věnovány těm nejvíce rozšířeným především v~osobních počítačích. Vysvětlím, k~čemu se dají vhodně využít a~jaké mají nedostatky. Následující přehled bude zároveň tvořit podklad pro~analýzu v~praktické části práce.

\section{Klávesnice}

Klávesnice je nejpoužívanější počítačovou periferií vůbec. Její princip je velmi jednoduchý: každé stisknutí či uvolnění tlačítka způsobí odeslání informace do~PC. Je možné detekovat události více tlačítek najednou, čehož využívají klávesové zkratky.

V~herním průmyslu se klávesnice využívá v~drtivé většině herních žánrů od~jednoduchých arkád, přes~závody až po~strategie. Jsou vhodné, pokud je potřeba rozlišit větší množství uživatelských vstupů, které mohou reprezentovány jednotlivými klávesami.

Klávesnice ale nemusí být vždy ideální volbou. Diskrétní zpracování vstupu (stisknuto, nestisknuto) představuje nepohodlí při~ovládání závodní hry: v~zatáčení je zhoršená citlivost, auto buď zatáčí naplno, nebo vůbec. Tento nedostatek se vývojáři snaží řešit například postupným natáčením kol, ale ani to není ideální. Při zatočení v~menší zatáčce je nutné přerušovaně uvolňovat klávesu, aby se vytvořila iluze mírně natočeného volantu. V~kombinaci s~myší může být nevýhodou horší dostupnost kláves vzdálenějších od~ruky.

\section{Myš}

Počítačová myš je druh polohovacího zařízení. Optický či laserový snímač detekuje pohyb myši po~podložce a~převádí jej na~pohyb kurzoru na obrazovce. Dále bývá myš vybavena několika ovládacími tlačítky.

Při~hraní má široké uplatnění tam, kde je využíván klasický kurzor nebo při~nutnosti souvislého, ale přesného pohybu jako například otáčení hráče v~FPS hře\footnote{First-person shooter - akční hra zobrazená z~pohledu herní postavy}.

Nevýhodou při~používání myši je jednostranná zátěž. Kvůli pohybu po~stole hráči zatěžují ruku s~myší více, než ruku na~klávesnici.

\section{Gamepad}

Gamepad je čistě herní periferie. Je to ovladač tvarovaný pro~použítí oběma rukama. Nachází se na~něm množství tlačítek a~může být doplněn jedním, nebo dvěma analogovými joysticky. Některé verze nabízejí i vibrační odezvu. Nejdříve se využíval u~herních konzolí, s~rozvojem osobních PC se však stal i~zde velmi populární.

Primarní zaměření na~hry dělá z~gamepadu vynikající ovladač pro~mnoho herních žánrů. Eliminuje problém ergonomie myši a~klávesnice a~všechna tlačítka jsou snadno dostupná. Proto se stal velmi oblíbeným.

Z~hlediska intuitivního ovládání gamepad zaostává podobně jako myš a~klávesnice. Hry jsou sice s~ním dobře ovladatelné, avšak často je potřeba tréninku pro~zvládnutí složitějších principů ovládání.

\section{Joystick a volant}

Joystick a~volant představují ovladače pro~specifičtější druhy her, na~druhou stranu mmohem věrněji simulují realitu.

Základní částí joysticku je páka umístěná kolmo v~pohyblivém kloubu. Má nejlepší uplatnění v~leteckých simulátorech, kde náklon páky mění polohu leteckých klapek. Ovládání je intuitivní a~snaží se přiblížit k~realitě.

Účelem volantu je simulace ovládání závodního vozu. Obvykle je dodáván s~dvěma nebo třemi pedály a~případně řadicí pákou. Vyšší modely mají vibrační odezvu, při vyjetí z~vozovky tak hráč cítí haptickou odezvu.

Joystick a~volant pomáhají věrně simulují letecké nebo~závodní prostředí, pro~které jsou určeny. Na~rozdíl od~předešlých periferií jsou ale použitelné pouze v~úzké oblasti her.

\section{Dotyková obrazovka}

Dotyková obrazovka na~osobních počítačích není v~současné době masově využívána. Větší rozšíření má v~oblasti mobilních zařízení, jako jsou mobilní telefony a~tablety, proto se více problematice dotykové obrazovky budu věnovat v~sekci \ref{section:touchscreen}. Některé hry však využívají dotykovou obrazovku jako~náhradu za~jiné periferie (např. myš). Záleží pak na~návrhu samotné hry, zda toto ovládání přinese nějaké benefity, či~bude spíše překážkou.

\section{Ovládání pohybem}

Kromě tradičních hardwarových periferií existují další technologie, jak~ovládat hry, a~to~pomocí přirozeného pohybu člověka. Tyto systémy snímají gesta a~pohyby hráče nebo ovladače a~podle nich vypočítají reakci ve~hře. Takovéto ovládání bývá snadné na~naučení, protože navozuje pocit přirozené reakce na~podněty herního prostředí.

V~této kapitole zmíním dvě technologie: motion capture (neboli snímání pohybu těla) a~akcelerometr. Obě tyto technologie silně rozšířují možnosti interakce s~elektronickými zařízeními, fungují však na~odlišném principu.

\subsection{Akcelerometr}

Akcelerometr je elektromechanické zařízení, které měří zrychlení sil ve~3 osách. Tyto síly mohou být statické, jako například tíhová síla, nebo dynamické - způsobeny pohybem nebo vibrováním akcelerometru.\cite{acc} Tato součástka umožňuje detekovat natočení zařízení v~prostoru a~jeho přibližný pohyb.

V~herním průmyslu se využívá především v~mobilních zařízeních a~v~herních konzolích. Může nahradit joystick, emulace volantu nebo polohovacího zařízení. 

V~případě, že~jako vstupní zdroj informace je poloha zařízení, může v~nevhodném prostředí docházet k~rušení: například při~jízdě v~autobuse je téměř nemožné hrát závodní hru, jelikož při zatáčení autobusu bude síla působící na~akcelerometr vychýlená a~bude tak docházet k~nechtěnému zatáčení.

Další informace o~pohybovém senzoru jsou uvedeny v~sekci \ref{section:accelerometer}.

\subsection{Motion capture}
\label{section:motion_capture}

Velmi zajímavou technologií je ovládání pohybem, tzv. \textit{motion capture}. Jedná se o~snímání části nebo~celého těla a~jeho převod v~reálném čase na~digitální model. Ten se využívá mimo jiné také k~ovládání hry. Nejvýraznějším ovladačem v~herním průmyslu se stal \textit{Kinect} vyvinutý v~roce 2009 firmou Microsoft. \cite{meetthekinect}

Motion capture se hodí například pro~sportovní nebo~akční hry a~to i~při více hráčích. Nevýhodou může být nutnost poměrně velkého prostoru pro~potřeby hraní.

\section{Srovnání}

V~minulých kapitolách jsem stručně popsal několik způsobů, jakým lze ovládat hry. Každý znich má řadu výhod i~nevýhod. Následující tabulka obsahuje shrnutí těchto ovladačů a~porovnání jejich možností. Vzhledem k~tématu této práce se~zde budu zabývat především tím, jak je daný způsob interaktivní a~zda tak může obohatit zážitek ze~hry.

\begin{table}[h]
\caption{Srovnání herních ovladačů}
\begin{tabularx}{\textwidth}{|l|X|X|}
\hline
\textbf{Ovladač} & \textbf{Výhody} & \textbf{Nevýhody} \\ \hline
Klávesnice & - počet kláves \newline - široké využití & - pouze dva stavy tlačítka (stisknuto, nestisknuto) \newline - některé klávesy hůře dostupné \\ \hline
Myš & - přenost & - jednostranná zátěž \\ \hline
Gamepad & - ergonomický \newline - navržen pouze pro~hry \newline - interaktivní vibrační odezva & - ovládání není intuitivní, je třeba se jej naučit \\ \hline
Joystick, volant & - věrně simuluje realitu \newline - interaktivní vibrační odezva & - vhodné jen pro~některé herní žánry \\ \hline
Dotyková obrazovka & - široké využití \newline - provázanost grafiky s~ovládáním & - v~některých hrách neúspěšně simuluje jiné periferie \\ \hline
Akcelerometr & - intuitivní & - rušení např. v~dopravním prostředku či~jiném pohybu \\ \hline
Motion capture & - velmi přirozené ovládání & - potřeba většího prostoru \\ \hline

\end{tabularx}
\end{table}


\chapter{Interaktivní prvky v mobilních telefonech}

V~první kapitole jsem představil nejčastější herní ovladače, které se využívají v~herních konzolích a~osobních počítačích. V~následující kapitole se budu věnovat analýze mobilních telefonů a~jejich senzorů interagujícíh s~okolním prostředím. Popíšu principy fungování a~budu zkoumat jejich herní využití jak v~mobilních hrách, tak jako interaktivní ovladač pro~počítačovou hru. Některé z těchto senzorů pak využiji při~tvorbě interaktivního ovladače v~kapitole~\ref{chapter:implementation}.

Senzory se budu snažit využít tak, aby mobilní telefon nesloužil pouze jako vstupní zařízení, a~proto se jej budu snažit provázat s~hrou samotnou. To umožní rozšířit herní realitu, kdy se ovladač stává neoddělitelnou součástí samotné hry. Zároveň to nabízí nové možnosti herních principů, které by na~tradičních periferiích nebyly realizovatelné.

\section{Dotykový displej}
\label{section:touchscreen}

Dotykový displej je označení pro kombinaci zobrazovacího displeje a průhledné vrstvy pokrývající displej, který je schopnej zaznamenávat dotyky vnějších těles, obvykle prstu nebo stylusu. \cite{gray2013does} Existuje několik technologií snímání dotyku, stručně popíšu ale pouze nejrozšířenejší technologi v dnešních mobilních zařízeních: kapacitní dotykový panel.

Kapacitní dotykový panel využívá elektrické vodivosti vnějších těles. Dotyk takového tělesa s povrchem displeje naruší elektrostatické pole obrazovky a jeho lokaci pak dále zpracuje řadič. Proto kapacitní displeje fungují pouze s vodivými předměty jako je lidský prst nebo speciální stylus.

Dotykové displeje nabízejí velmi široké možnosti uplatnění ve hrách. Na rozdíl od klasických ovladačů totiž propojují zobrazení grafiky se samotným ovládáním. Například přesun objektů v herním světě je velmi přirozený: hráč se jednoduše dotkne objektu a tahem prstu přesune po obrazovce. Ve srovnání s myší pak odpadá \textit{mezivrstva} ve formě periferie připojené k zařízení a hráčův prst se tak stává samotným kurzorem.

Možnosti dotykových displejů, jako jsou gesta a snímání více dotyků najednou, zásadním způsobem zjednodušily ovládání mobilních telefonů a tedy i her. Některé hry jsou přímo postaveny na využití dotykových displejů, jednou z nich je velmi úspěšná hra \textit{Fruit Ninja}\cite{fruitninja}. Hlavním ovládacím prvkem je tah prstem, který simuluje seknutí čepelí. Hráč se snaží jedním seknutím přeseknout co největší množství letícího ovoce najednou. Hra je příkladnou ukázkou využití dotykové obrazovky: letící ovoce je podnětem, aby se hráč trefil do daného místa na obrazovce. Hra a ovládání tak splývají v jednu ucelenou část.

\section{Akcelerometr}
\label{section:accelerometer}


\section{Gyroskop}

\section{Senzor přiblížení}

Posledním zde uvedeným senzorem je senzor přiblížení, nebo také proximity senzor. Jedná se o součástku fungující na principu detekce elektromagnetického záření. \cite{proximity}

\chapter{Tvorba systému}
\label{chapter:implementation}

\section{Analýza}
\section{Požadavky}
\section{Návrh}
\section{Implementace}
\section{Testování}
\section{Dokumentace}
\section{Možnosti rozšíření}

\begin{conclusion}
	%sem napište závěr Vaší práce
	
	%TODO: možnosti rozšíření %
\end{conclusion}

\bibliographystyle{csn690}
\bibliography{mybibliographyfile}

\appendix

\chapter{Seznam použitých zkratek}
% \printglossaries
\begin{description}
	\item[GUI] Graphical user interface
	\item[XML] Extensible markup language
\end{description}

\chapter{Obsah přiloženého CD}

%upravte podle skutecnosti

\begin{figure}
	\dirtree{%
		.1 readme.txt\DTcomment{stručný popis obsahu CD}.
		.1 exe\DTcomment{adresář se spustitelnou formou implementace}.
		.1 src.
		.2 impl\DTcomment{zdrojové kódy implementace}.
		.2 thesis\DTcomment{zdrojová forma práce ve formátu \LaTeX{}}.
		.1 text\DTcomment{text práce}.
		.2 thesis.pdf\DTcomment{text práce ve formátu PDF}.
		.2 thesis.ps\DTcomment{text práce ve formátu PS}.
	}
\end{figure}

\end{document}
